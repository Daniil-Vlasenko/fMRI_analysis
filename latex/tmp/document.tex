% !TeX spellcheck = ru_RU
\documentclass[specialist,
substylefile = spbu_report.rtx,
subf,href,colorlinks=true, 12pt]{disser}

\usepackage{amsmath,amssymb,amsthm,amscd,amsfonts, mathtools}
\usepackage{multirow}
\usepackage{float}
\usepackage[utf8]{inputenc}
\usepackage[a4paper,
left=2cm, right=1.5cm, top=2cm, bottom=2cm]{geometry}
\usepackage{graphicx}
\usepackage[russian]{babel}
\usepackage{xcolor}
\usepackage{float}

\makeatletter
\newcommand*{\rom}[1]{\expandafter\@slowromancap\romannumeral #1@}
\newcommand\setItemnumber[1]{\setcounter{enum\romannumeral\@enumdepth}{\numexpr#1-1\relax}}
\makeatother


\setcounter{tocdepth}{3}
\setcounter{secnumdepth}{3}


\newtheorem{defenition}{Определение}
\newtheorem*{purpose*}{Цель работы}
\newtheorem*{prob_task*}{Вероятностная постановка задачи классификации}
\newtheorem*{algo_task*}{Алгоритмическая постановка задачи классификации}
\newtheorem*{prob_def*}{Вероятностное определение $w_{ij}^k$}
\newtheorem*{algo_def*}{Алгоритмическое определение $w_{ij}^k$}

\begin{document}
	\institution{%
		Санкт-Петербургский государственный университет\\
		Прикладная математика и информатика
	}
	
	\title{Отчет по Научно-исследовательской работе}	
	\topic{Задачи классификации мозговой активности при помощи синолитических сетей}	
	\author{Власенко Даниил Владимирович}
	\group{группа 19.Б04-мм}
	\sa{Шпилёв Пётр Валерьевич\\
		Кафедра Статистического Моделирования}
	\sastatus {к.\,ф.-м.\,н., доцент}	
	\city{Санкт-Петербург}
	\date{\number\year}	
	\maketitle
	\tableofcontents
	

	
	\chapter{Модель}	
	Множество фМРТ будем обозначать $\Omega$, а множество режимом мозговой активности~--- $\Sigma = \{$\rom{1}, \rom{2}$\}$. $(\widetilde{\Omega}, \widetilde{\Sigma}) =  \{(\omega_{n}, \sigma_{n})\}_n$ --- конечная выборка из $(\Omega, \Sigma)$, необходимая для построения и обучения модели.				
	
	\section{Построение графа на основе фМРТ} 			
	Сначала фМРТ данные конвертируется в четырехмерный массив~$a$. Первые три индекса $x$, $y$, $z$ фиксируют положение вокселя фМРТ, а четвертый индекс $t$ отвечает за время. Таким образом через $a_{xyzt}$ будем обозначать значение конкретного вокселя в конкретный момент времени, а через $a_{xyz}$ будем обозначать все значения вокселя с индексами $x$, $y$, $z$. Иногда будет удобно использовать для индексации конкретного вокселя не три целых числа, а одно. Для этого положим, все воксели в пространстве проиндексированы натуральными числами.		
	
	На основе массив $a$ в дальнейшем будет строится граф $g = (V, E, R, W)$, где $V = \{v_i\}_i$~--- множество вершин, $E = \{e_{ij}\}_{ij}$~--- множество неориентированных ребер, $R = \{r_i\}_i$~--- множество значений вершин, $W = \{w_{ij}\}_{ij}$~--- множество весов ребер.
	
	Обсудим как вычисляются значения вершин $R$. Каждая вершина графа отражает собой конкретный воксель и у нее есть значение --- некоторое неотрицательное действительное число, но воксель это временной ряд с множеством значений. Для вычисления значения вершины в модели используется статистика $T$, которая будет преобразовывать все значения вокселя в одно число. Таким образом можно ввести новый трехмерный массив $a^{T} = T(a)$, т.е для $\forall x, y, z$ $a^{T}_{xyz} = T(a_{xyz})$. Значения массива $a^{T}$ и будут использоваться в качестве значений вершин $R$. Статистика $T$ будет выбираться исходя из результатов тестирования модели.
	
	
	------------\\\\\\\\
	
	
	Обсудим как вычисляются значения весов ребер $W$. Дадим вероятностное определение веса ребра.				
	\begin{equation}
		w_{ij} = P(y_k = \rom{2} | r_i, r_j) - P(y = \rom{1} | r_i, r_j)
		\label{eq:1}
	\end{equation}		
	Вес ребра $w_{ij}$ равен разнице вероятностей режимов работы мозга при условии значений инцидентных ребру вершин и принимает значения от $-1$ до $1$. Соответственно, если вес ребра $w_{ij} < 0$ , то ребро $e_{ij}$ несет в себе информацию о том, что более вероятно, что фМРТ было сканировано с мозга, который находился в режиме \rom{1}, а если вес ребра $w_{ij} > 0$, то данное ребро несет в себе информацию о том, что более вероятно, что фМРТ было сканировано с мозга, который находился в режиме \rom{2}. Чем больше вес ребра $|w_{ij}|$ тем больше информации для классификации несет в себе ребро $e_{ij}$.
	
	На практике для вычисления таких вероятностей используются вероятностные классификаторы $Cl_{ij}: \{y |(r_i, r_j), \{(r_i^n, r_j^n)\}_n, \{y_n\}_n\} \rightarrow [0, 1]$, которые обучаются на имеющейся выборке $(\widetilde{X}, \widetilde{Y})$. Таким образом для каждого ребра $e_{ij}$ требуется обучить свой вероятностный классификатор $Cl_{ij}$ для последующего вычисления весов ребер $W$. В данной работе используются вероятностные классификаторы, в основе которых лежит метод опорных векторов с радиально-базисным ядром.		
	
	Далее строиться граф-сетка, то есть такой граф, в котором каждый внутренний воксель связан с 26 своими соседями.
	
	Так как в фМРТ изображен не только мозг, но и пространство вокруг головы испытуемого, следует удалить из графа ребра, которые инциденты с вершинами, значения которых ниже порогового значения $r$. Такие ребра не несут полезной информации для классификации. Так же из графа следует удалить ребра, абсолютное значение веса которых ниже порогового значения $w$. Ребра, абсолютное значение веса которых близко к нулю, так же не несут полезной информации для классификации.
	
	После построения графа $g$ вычисляются его характеристики $\{f_u\}_u = \{F_u(g)\}_u$, таким образом от графа остается последовательность чисел $\{f_u\}_u$. Характеристики графа выбираются при тестировании модели.
	
	Далее на основе $\{f_u\}_u$ происходит итоговая классификация фМРТ данных $\omega$ с помощью классификатора $Cl$, который был обучен на выборке $\{\{f_u^n\}_u\}_n$. В качестве классификатора в работе использовался классификатор, основанный на методе опорных векторов с радиально-базисным ядром.

	
	
	
\end{document}
